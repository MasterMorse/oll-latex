%%%%%%%%%%%%%%%%%%%%%%%%%%%%%%%%%%%%%%%%%%%%%%%%%%%%%%%%%%%%%%%%%%%%%%%%%%%
%                                                                         %
%      This file is part of the 'openLilyLib' library.                    %
%                                ===========                              %
%                                                                         %
%              https://github.com/lilyglyphs/openLilyLib                  %
%                                                                         %
%  Copyright 2012-13 by Urs Liska, lilyglyphs@ursliska.de                 %
%                                                                         %
%  'openLilyLib' is free software: you can redistribute it and/or modify  %
%  it under the terms of the GNU General Public License as published by   %
%  the Free Software Foundation, either version 3 of the License, or      %
%  (at your option) any later version.                                    %
%                                                                         %
%  This program is distributed in the hope that it will be useful,        %
%  but WITHOUT ANY WARRANTY; without even the implied warranty of         %
%  MERCHANTABILITY or FITNESS FOR A PARTICULAR PURPOSE. See the           %
%  GNU General Public License for more details.                           %
%                                                                         %
%  You should have received a copy of the GNU General Public License      %
%  along with this program.  If not, see <http://www.gnu.org/licenses/>.  %
%                                                                         %
%%%%%%%%%%%%%%%%%%%%%%%%%%%%%%%%%%%%%%%%%%%%%%%%%%%%%%%%%%%%%%%%%%%%%%%%%%%

% Contributor's Guide - Tutorial part
%
%%%%%%%%%%%%%%%%%%%%%%%%%%%%%%%%%%%%%%%%%%%%%%%%%%%%%%%%%%%%%%%%%%%%%%%%%%%

\documentclass[openlilylib-cg]{subfiles}

\begin{document}

\parttitle[Urs Liska]{\LaTeX{} Usage}

\begin{authorAbstract}{Urs Liska}
To be done.\\
Describe the main books, and how the parts are included into them using the \package{subfile} approach.\\
Describe the documentclass(es), list the included packages
\end{authorAbstract}

\chapter{Directory structure, Document Class(es) etc.}

If you want to contribute work for \openlilylib{} we assume that you have aqcuired the complete directory by either cloning the Git repository from Github or downloading the complete archive.
For writing \LaTeX{} documents there are two subdirectories of interest: 
\begin{description}
\item[\dir{/documentation}] is the place where all documentation files are stored.
\item[\dir{/documentation/latex}] contains the basic document classes and packages used for authoring documents.
You have to make sure your \LaTeX{} installation finds this directory -- either by including it in the search path or by placing a link to it inside your \dir{texmf} tree.
You won't need to do anything in this directory if you are only going to write documents (except maybe investigating it).
\end{description}

\openlilylib{} documentation is organized using the \package{subfiles} package.
At the moment there are several main books on \openlilylib{} subjects.
They are located in subdirectories of \dir{/documentation}:
\begin{description}
\item[\dir{CG/openLilyLib\_CG.tex}] The Contributor's Guide, explaining how to author \LaTeX{} documents and participate in the development of the different libraries and packages.
\item[\dir{musicexamples/musicexamples.tex}] Documentation of \package{musicexamples}, a collection of a \LaTeX{} package, LilyPond resources and Python scripts to include and manage music examples in \LaTeX{} documents.
\item[\dir{OLLib/OLLib.tex}] OLLib is the short name for the core \openlilylib{} library of LilyPond tools.
\item[\dir{tutorials/tutorials.tex}] Independent tutorials, mainly on LilyPond topics.
Maybe they can one day form a somewhat contigious “book”, but we start it as a random collection of what users are willing to contribute.
Articles may range from short HowTos to full-fledged tutorials.\\
\textsf{\textbf{Important notice}}: If you are interested in contributing some knowledge about LilyPond but don't know how to express yourself in \LaTeX{} \emph{please} don't hesitate to contact us -- if it seems an interesting contribution we'd surely not mind helping you converting it to a useable format.
\end{description}

These books have the document class \package{OLLbook}, which is based on \package{scrbook} from \package{KOMA-Script}.
The document class implies the use of XeLaTeX as its processing engine because the \package{lilyglyphs} package relies on the \package{fontspec} package.

In order to compile the documents you have to have the preset OpenType fonts installed: Linux Libertine O, Linux Biolinum O (\url{http://www.linuxlibertine.org/})and Inconsolata (\url{http://www.levien.com/type/myfonts/inconsolata.html}).

Instead of setting a title you should use \cmd{setbookmaintitle\{\#1\}} in the document preamble.
This will set the title but will additionally keep it for (automatic) use on the title page of the sub-documents (see below).
Inside the \env{document} environment you should use \cmd{maketitle} as usual.

\bigskip
If you are going to provide an independent contribution (e.\,g.\ a tutorial) you will usually have to create a \package{subfile} to one of these books (if this hasn't been prepared for you already).
Such a file should be placed in a sensibly named subdirectory of the main book and be set up as follows:
\begin{quote}
\begin{verbatim}
\documentclass[../nameofmainbook]{subfiles}
\begin{document}
\parttitle{Title of Your Contribution}
\begin{authorAbstract}[Optional, Contributors]{Your Name}
\chapter{Your First Heading}

% Contents ...

\includelicensepart
\end{document}
\end{verbatim}
\end{quote}
The document class is \package{subfiles}, with the optional argument pointing to the main book's file (without extension).
Instead of a title you use the appropriate one of the new commands \cmd{parttitle}, \cmd{chaptertitle} or \cmd{sectiontitle}
They are quite fancy and behave differently if you compile the subfile or the enclosing main document.
If compiled as part of the main book the simply insert a \cmd{part}, \cmd{chapter} or \cmd{section} command in the document.
But if you compile your subfile directly they will do several things:

Firstly they will print a special version of a title page -- which will be the title page of your standalone document.
This contains the title of your bookpart (the argument to the above commands), but without a section number.
They create a header indicating that the document is part of the \openlilylib{} documentation and what book it is taken from.
This works only if \cmd{setbookmaintitle} is placed correctly in the preamble of the main book -- it this isn't the case you will see an informative message instead.
At the bottom of the page you will find the copyright notice.
This is printed automatically and only if the subfile is compiled directly.
There is also the explicit command \cmd{copyrightnotice} that also only prints something when called from a subfile.
This is done to prevent the main books to be spammed by identical copyright notices for each and every subfile, while allowing them to be included in the standalone documents.

After the \cmd{parttitle} or the first section heading you should place the \env{authorAbstract} environment to credit your contribution (and potential other contributors)

At the very end of your subfile you should place one of the commands\cmd{includelicensepart}, \cmd{includelicensechapter} or \cmd{includelicensesection}.
These commands insert the appropriate heading and input the file with the complete GPL and FDL, but only if we are inside a subfile.
So standalone documents can also include the full license texts.


\chapter{Command Reference}
\begin{authorAbstract}{Urs Liska}
This is the reference on specific commands that have been implemented for \openlilylib's \LaTeX{} documentation.
It is somewhat sketchy, but I want all commands at least to be enumerated as soon as they are implemented.
\end{authorAbstract}

\section{Formatting}
One should \emph{never} apply manual formattings but always semantic markup through the use of commands.
If there is no suitable command available then an author should define one himself or ask for new ones.

\subsection{Environments}
\begin{description}
\item[\env{authorAbstract}] Although \openlilylib{} is a collaborative effort, we want to attribute the original author of tutorials or noteable chunks of documentation.
This environment should be used after a sectioning command.
The original author is given as the mandatory argument to the environment.
Contributors may be given as an optional argument (use a comma-separated list of \cmd{contributor} items).
The contents of the environment is formatted as an abstract.
\todo{It is intended to implement an index of original authors and their contributions.}\\
If you want to mark a substantial contribution in a place where an abstract seems inappropriate use the command \cmd{originalAuthor} instead.
\item[\env{knownIssues}] Use this environment to print a section with -- well -- known issues and warnings \dots
\end{description}

\subsection{Paragraph Formatting}
\begin{description}
\item[None yet]
\end{description}

\subsection{Character Formatting}

\begin{description}
\item[\cmd{cmd}] Commands that start with a backslash can be printed using this command. It switches the font and prints the backslash in front of the command name.
\item[\cmd{dir}] Directories and filenames can be highlighted with this command. 
\item[\cmd{env}] Environment names are highlighted using this command.
\item[\cmd{package}] You can enter package names with this command.
\item[\cmd{todo}] Any TODO item should be entered in the source using this command, because it will guarantee a consistent appearance and it's quite spottable.
This may someday be enhanced to create an index.
\end{description}


\section{Other Commands}
\begin{description}
\item[\cmd{contributor}] Use this command together with \cmd{originalAuthor} or the \env{authorAbstract} environment.
So far this is only a formatting command, but it is intended to create an index of contributors and their contributions (see \ghIssue{7})
\item[\cmd{ghIssue}] Use this command to create a link to an issue report on \openlilylib's old Github site.
Pass the name of the repository (e.\,g.\ “musicexamples”) as the first and the plain issue number (without leading zeroes or other entities) as the second argument.
\item[\cmd{ollTicket}] Use this command to create a link to an issue report on \openlilylib's new sourceforge.net tracker.
Pass the plain issue number (without leading zeroes or other entities) as the mandatory argument.
\item[\cmd{openlilylib}] Use this command to print \openlilylib's name.
This may be formatted more stylishly later \dots
\item[\cmd{originalAuthor}] This is a stripped down command version of the \env{authorAbstract} environment.
You can use this command to indicate authorship for some piece of text without the need for providing an abstract.
The mandatory argument is the author's name, and you may supply a comma-separated list of \cmd{contributor} entries as the optional argument.
\end{description}

\section{Music and Code Examples}
Inserting music examples can be achieved using \package{musicexamples}.
This package is part of \openlilylib{} and is included by the document class.
Please refer to the package documentation for how to work with it.

\todo{Code listings (LilyPond or \LaTeX) haven't been implemented yet}

\includelicensechapter

\end{document}
